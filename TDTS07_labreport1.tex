
%%% Local Variables: 
%%% mode: latex
%%% TeX-master: t
%%% End: 

\documentclass[12pt]{article}
\usepackage{graphicx}
\usepackage[T1]{fontenc}		% F�r svenska bokst�ver
\usepackage[swedish]{babel}
\usepackage{listings}


\title{TDTS07 Lab Report 1}
\author{Nora Bj�rklund and Christopher Hallberg}
\begin{document}
\maketitle

\section{Running the Simulation}
All necessary code is submitted together with this report.
To compile the code use the included makefile. Then to start the
simulation run ''./trafficlight.x t input output'', where ''t'' is
the simulation time in seconds, ''input'' the file specifying arriving
cars and ''output'' the file where the results of the simulation is stored. 

\section{Traffic-light}
The traffic light consists of four sensors which are connected to a
controller that can at the right moment give a green signal on a
light where a car is waiting. The cars are generated from an input
file which is read by the car generator, input\_gen. Which car is
granted green light is read by the monitor that writes to an output
file.
\subsection{Input Generator}
The input generator simply reads a text document and inputs cars from
north, south, east or west. The format of the input file is the
letters W, E, S, or N followed by a space and then a time value in
seconds and finally completed with a newline. The letter indicates 
from what direction a car arrives and 
the time value indicates how many seconds after the previous car a new 
one will be generated.
\subsection{Sensor}
The sensor detects when a car have arrived at its trafficlight and
sends a signal to the controller that it has a car waiting. There is
one sensor per direction.
The sensors has an SC\_THREAD with an infinite loop checking if there is a new
car every 50 millisecond, any signal shorter than that is considered a
glitch. If it detects a new car it raises an event
which is detected by an SC\_METHOD. This method inverts an internal
signal and puts it on the output for the controller. That is, both
falling and rising edge of the signal between the sensors and
controller indicates a new car.
\subsection{Controller}
The controller is the unit which decides what directions should be
green. It is connected to the four sensors from which it receives
signals telling it if new cars have arrived. There are four
SC\_METHODs, one for each direction, which are sensitive to signals
from the respective sensors. When a car is detected by one of the
methods a variable is set to indicate that a car is waiting at a
specific light. 

To calculate which lights should be green an SC\_THREAD consisting of
an infinite loop is used. The thread first checks if there are waiting
cars from west or east, then sets the light to green for the
correct direction. Both lights are switched to green if they both have
waiting cars, if only one direction have cars then only that direction
will be switched to green. The light will be green for three seconds,
after which the lights are set to red and the waiting car variables
are reset to false. After checking east and west the thread then moves
on to do the same thing for north and south.

With this it can be seen that the lights work independently and that
orthogonal directions cannot have green at the same time. Since all
directions are checked for every loop iteration there is no
starvation, a car will eventually get green light. It can also be seen
that opposite directions can have green light at the same time if
there are cars waiting when the light switches. However, if for
example a car arrives from the north direction while the south
direction has green light, it has to wait for the next iteration. 

\subsection{Monitor}
The monitor uses an SC\_METHOD sensitive to when the controller changes
the status of the light. When a light is changed it reads the status
of all the lights and prints it to an output file together with a time
stamp. 

\subsection{Traffic-light Testbench}
The testbench is where the sc\_main is and where the different
components are instantiated and connected together. 

\section{Tests}
To verify that the traffic light is correct different simulations
have been executed.
\subsection{Independence}
This test is to verify that the opposite directions works
independently of each other. To do this only one car from south is
generated as can be seen in \ref{lst:indinput}. Note that the break
is only there to prevent the last input line to be read twice which
would be the case otherwise. Also, the breaking works with anything that
is not N, S, W or E. 
\lstinputlisting[caption={Independent Test Input},label={lst:indinput}]{ind_input}
The output of the monitor is unfortunately too wide to fit nicely in
this document, but it can be seen in the file ''ind\_output'' submitted
together with this document. Every time the a light switches the
current status of all lights are printed on one line, ''0'' means red and
''1'' means green. It is clear that after 4.1 s only the south light turns
green for three seconds. Hence, the light work independently.

\subsection{Safety}
This test verifies that the light cannot be green at the same time for
two orthogonal directions. \ref{lst:safeinput} was used as input
for the test. The output file is the one called ''safe\_output''.
\lstinputlisting[caption={Safety Test Input},label={lst:safeinput}]{safe_input}
The first car arrives from the north after one second, then directly
after a car arrives from the east. Five seconds later a car arrives
from the west at the same time as cars from both north and
south. Looking in the output file it is clear that the orthogonal
directions never have green light at the same time. As a side-note
when cars from orthogonal directions arrive at the same time, the ones
from north and south will have precedence because of the way the
controller is written. 

\subsection{Starvation}
To make sure that a car arriving at a light will eventually get green
light \ref{lst:starinput} was used. The result of the test is saved in the output
file called ''star\_output''. 
\lstinputlisting[caption={Starvation Test Input},label={lst:starinput}]{star_input}
In this test there is a single car from the north in a steady stream
of cars from the west. In the output file it can be seen that the car
from the north is not starved and does not have to wait for a long
time before its light turns green.
 
\end{document}